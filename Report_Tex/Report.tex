\documentclass[a4paper]{article}

\usepackage[utf8]{inputenc}
\usepackage[T1]{fontenc}
\usepackage[english]{babel}


\usepackage[style=numeric, backend=biber]{biblatex}
\addbibresource{bibliography.bib}

%opening
\title{RoboCup Final Project Report \\ Color Detection }
\author{Jakob Stimpfl \and Charlotte Burmeister}

\begin{document}

\maketitle
\tableofcontents
%\begin{abstract}

%\end{abstract}



\section{Introduction}
The Nao robot is a humanoid robot produced by Aldebaran. It is 574 mm high and has xx joints, controlled by stepper motors.

In this project to goal was to enable the robot to listen to a color, interpret the color heard as RGB values and then point to this color.



\section{Implementation}

As predetermined by the course the Python Software Development Kit (SDK) was used.
This SDK provides the use of C++ modules and enables the user to create own Python modules.\cite{API}

\subsection{Speech Detection}
\subsection{Color Detection and Movement}
\section{Conclusion}

\printbibliography

\end{document}
